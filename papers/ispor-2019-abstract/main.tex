\documentclass{article}


\usepackage{arxiv}

\usepackage[utf8]{inputenc} % allow utf-8 input
\usepackage[T1]{fontenc}    % use 8-bit T1 fonts
\usepackage{hyperref}       % hyperlinks
\usepackage{url}            % simple URL typesetting
\usepackage{booktabs}       % professional-quality tables
\usepackage{amsfonts}       % blackboard math symbols


\title{TRANSPARENT META-ANALYSES ENABLED BY BLOCKCHAIN TECHNOLOGY}

\author{
  Jaclyn Hearnden \\
   \And
  Justin D.~Harris \\
  Microsoft Research\\
  Montreal, Canada \\
  \texttt{justin.harris@}
}

\begin{document}
\maketitle

% keywords can be removed
\keywords{Meta-anlysis \and Blockchain \and Reproducibility}


\section{OBJECTIVES}
\label{sec:objectives}
Blockchain applications have proven useful in clinical research and healthcare to improve transparency and accountability.
These same principles can be applied to meta-analyses involving clinical trial data to ensure the quality of health economics and outcomes research.


\section{METHODS}
\label{sec:methods}
Smart contracts are code that exist on a blockchain, they are typically designed to publicly disclose the result of a computation.
We propose the use of a smart contract aligned with standard guidelines to compute the results of meta-analyses.
Toward this goal, clinical trial publication data are stored in easily accessible tables (or, hash maps), each uniquely identified.
Later analyses, each in their own smart contract, refer to these publications by their unique identifiers and associated fields.
The results of all analyses stemming from clinical trial publications are contributed to a master smart contract as a new entry along with the original data.
If a mistake is made in the variables or computation of a given meta-analysis, then analyses that depend on it can be easily updated by the creation of this network linking the reviews and studies by dependencies.
Decentralization of this network, as mandated by blockchain technology, ensures that no body is responsible for its maintenance and numerous parties may validate the computations.
Further, this structure favors transparency, and thereby helps to expose biases that may affect comparative-effectiveness research.

\section{CONCLUSIONS}
\label{sec:conclusions}
The number of clinical trials posting results per year has climbed to over 36,000 \cite{ClinicalTrialsTrends}.
Tracking and combining their successive analyses and results demands a decentralized network of trust, as we propose would be supported by blockchain technology. 

\bibliographystyle{unsrt}  
\bibliography{references}

\end{document}
